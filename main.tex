\def\lncs{0}


\ifnum\lncs=0
  \documentclass[11pt]{article}
  \usepackage[top=3cm, bottom=3cm, left=2cm, right=2cm]{geometry}      % [top=2cm, bottom=2cm, left=2cm, right=2cm]
  \geometry{letterpaper}                   % ... or a4paper or a5paper or ...
  %\geometry{landscape}                % Activate for for rotated page geometry
  %\usepackage[parfill]{parskip}
  \usepackage{amsthm}
  \newtheorem{theorem}{Theorem}
  \newtheorem{lemma}[theorem]{Lemma}
  \newtheorem{proposition}[theorem]{Proposition}
  \newtheorem{corollary}[theorem]{Corollary}
  \newtheorem{conjecture}{Conjecture}
  \newtheorem{definition}{Definition}
  \newtheorem{assumption}{Assumption}
  \newtheorem{claim}{Claim}
  \newtheorem{problem}{Problem}
  \newtheorem{construction}{Construction}
  \newtheorem{remark}{Remark}
  \renewcommand{\vec}[1]{\vect{#1}}      % activate for lncs style vectors

\else
  \documentclass[a4paper,UKenglish,cleveref, autoref, thm-restate]{lipics-v2021}
  \newtheorem{construction}{Construction}
  \renewcommand{\paragraph}[1]{\subsubsection{#1}}
\fi

\usepackage[dvipsnames]{xcolor}

\usepackage{tikz}
\usepackage{graphicx}
\usepackage{amssymb, amsmath, amsfonts}
\usepackage{enumerate}
\usepackage{hyperref}
\usepackage{xspace}
\usepackage{graphicx}
\usepackage{latexsym}
\usepackage{color}
\usepackage{framed}
\usepackage{algpseudocode}
\usepackage{breakcites}

\mathchardef\mhyphen="2D
\newcommand{\secref}[1]{\mbox{Section~\ref{#1}}}
\newcommand{\subsecref}[1]{\mbox{Subsection~\ref{#1}}}
\newcommand{\apref}[1]{\mbox{Appendix~\ref{#1}}}
\newcommand{\thref}[1]{\mbox{Theorem~\ref{#1}}}
\newcommand{\exref}[1]{\mbox{Example~\ref{#1}}}
\newcommand{\defref}[1]{\mbox{Definition~\ref{#1}}}
\newcommand{\corref}[1]{\mbox{Corollary~\ref{#1}}}
\newcommand{\lemref}[1]{\mbox{Lemma~\ref{#1}}}
\newcommand{\assref}[1]{\mbox{Assumption~\ref{#1}}}
\newcommand{\probref}[1]{\mbox{Problem~\ref{#1}}}
\newcommand{\clref}[1]{\mbox{Claim~\ref{#1}}}
\newcommand{\propref}[1]{\mbox{Proposition~\ref{#1}}}
\newcommand{\remref}[1]{\mbox{Remark~\ref{#1}}}
\newcommand{\consref}[1]{\mbox{Construction~\ref{#1}}}
\newcommand{\figref}[1]{\mbox{Figure~\ref{#1}}}
\newcommand{\conditionalpara}[1]{
\ifnum\lncs=1\noindent \textbf{#1} \else \paragraph{#1}\fi}
\newcommand{\conditionaleqn}[1]{\ifnum\lncs=1$#1$\else \[#1\]\fi}
\DeclareMathOperator*{\expe}{\mathbb{E}}
\DeclareMathOperator*{\var}{\text{Var}}
\DeclareMathOperator*{\Exp}{\mathbb{E}}
\DeclareMathOperator*{\argmax}{arg\,max}
\DeclareMathOperator*{\argmin}{arg\,min}

% Commands
\newcommand{\comment}[1]{\textcolor{Mulberry}{#1}}
\newcommand{\todo}[1]{\textcolor{red}{TODO: #1}}
\newcommand{\here}{\todo{Stopped here! }}

%Probability
\newcommand{\Ex}[1]{\Exp\left[#1\right]}
\newcommand{\Exlim}[2]{\Exp\limits_{#1}\br{#2}}
\newcommand{\Prob}[1]{\Pr\br{#1}}
\newcommand{\Problim}[2]{\Pr\limits_{#1}\br{#2}}
\newcommand{\p}[1]{p\prns{#1}}

% Entropy
\newcommand{\ent}[1]{H\prns{#1}}
\newcommand{\minent}[1]{H_{\infty}\prns{#1}}
\newcommand{\acminent}[2]{\overset{\sim}{H}\!_{\infty}\prns{#1\,|\,#2}}
\newcommand{\hart}[1]{H_0\prns{#1}}


%MACROS galore!!!
\newcommand{\class}[1]{{\ensuremath{\mathsf{#1}}}}
\newcommand{\gen}{\ensuremath{\class{Gen}}\xspace}
\newcommand{\aux}{\ensuremath{\class{Advise}}\xspace}
\newcommand{\advise}{\ensuremath{\class{advise}}\xspace}
\newcommand{\rep}{\ensuremath{\class{Rep}}\xspace}
\newcommand{\sketch}{\ensuremath{\class{SS}}\xspace}
\newcommand{\rec}{\ensuremath{\class{Rec}}\xspace}
\newcommand{\enc}{\ensuremath{\class{Enc}}\xspace}
\newcommand{\wt}{\ensuremath{\textsf{wt}}\xspace}
\newcommand{\dec}{\ensuremath{\class{Dec}}\xspace}
\newcommand{\prg}{\ensuremath{\class{prg}}\xspace}
\newcommand{\zo}{\ensuremath{\{0, 1\}}}
\newcommand{\vect}[1]{\ensuremath{\mathbf{#1}}}
\newcommand{\zq}{\ensuremath{\mathbb{Z}_q}}
\newcommand{\Fq}{\ensuremath{\mathbb{F}_q}}
\newcommand{\sample}{\ensuremath{\class{Sample}}\xspace}
\newcommand{\neigh}{\ensuremath{\class{Neigh}}\xspace}
\newcommand{\error}{\ensuremath{\class{Err}}\xspace}
\newcommand{\weight}{\ensuremath{\class{Wgt}}\xspace}
\newcommand{\dis}{\ensuremath{\mathsf{dis}}}
\newcommand{\bin}{\ensuremath{\mathsf{Bin}}}
\newcommand{\decode}{\ensuremath{\mathsf{Decode}}}
\newcommand{\guess}{\mathsf{guess}}
\newcommand{\nullsp}{\mathtt{null}}


\newcommand{\A}{\mathcal{A}}
\newcommand{\F}{\mathbb{F}}


\newcommand{\metric}{\ensuremath{\mathtt{Metric}}\xspace}
\newcommand{\bdde}{\ensuremath{\mathtt{BDDE}}\xspace}
\newcommand{\bdders}{\ensuremath{\mathtt{BDDE-RS}}\xspace}
\newcommand{\bdderl}{\ensuremath{\mathtt{BDDE-RL}}\xspace}
\newcommand{\findrep}{\ensuremath{\mathtt{FIND-REP}}\xspace}
\newcommand{\hill}{\ensuremath{\mathtt{HILL}}\xspace}
\newcommand{\hillrlx}{\ensuremath{\mathtt{HILL\mhyphen rlx}}\xspace}
\newcommand{\yao}{\ensuremath{\mathtt{Yao}}\xspace}
\newcommand{\unp}{\ensuremath{\mathtt{unp}}\xspace}
\newcommand{\unprlx}{\ensuremath{\mathtt{unp\mhyphen rlx}}\xspace}
\newcommand{\metricstar}{\ensuremath{\mathtt{Metric}^*}\xspace}
\newcommand{\metricd}{\ensuremath{\mathtt{Metric}^*\mathtt{-d}}\xspace}
\newcommand{\hillstar}{\ensuremath{\mathtt{HILL}^*}\xspace}
\newcommand{\hillprime}{\ensuremath{\mathtt{HILL'}}\xspace}
\newcommand{\metricprime}{\ensuremath{\mathtt{Metric'}}\xspace}
\newcommand{\metricprimestar}{\ensuremath{\mathtt{Metric'}^*}\xspace}
\newcommand{\hillprimestar}{\ensuremath{\mathtt{HILL'}^*}\xspace}
\newcommand{\poly}{\ensuremath{\mathtt{poly}}\xspace}
\newcommand{\rank}{\ensuremath{\mathtt{rank}}\xspace}
\newcommand{\ngl}{\ensuremath{\mathtt{ngl}}\xspace}
\newcommand{\Hoo}{\mathrm{H}_\infty}
\newcommand{\Hav}{\tilde{\mathrm{H}}_\infty}
\newcommand{\Hfuzz}{\mathrm{H}^{\mathtt{fuzz}}_{t,\infty}}
\newcommand{\Huse}{\mathrm{H}_{\mathtt{usable}}}
\newcommand{\Dom}{\mathsl{Dom}}
\newcommand{\Range}{\mathsl{Rng}}
\newcommand{\Keys}{\mathsl{Keys}}
\def\col{\mathrm{Col}}




% Useful Notation
\newcommand{\defined}{:=}
\newcommand{\sbr}[1]{ \!\left\{ #1 \right\} }
\newcommand{\br}[1]{\!\left[#1\right]}
\newcommand{\prns}[1]{\!\left(#1\right)}
\renewcommand{\log}[1]{\mathsf{log}\prns{#1}}
\newcommand{\zeroone}[1]{\sbr{0,1}^{#1}}

\title{A Negative Result for Fuzzy Extractors}
\author{Luke Demarest\\University of Connecticut\\luke.h.demarest@gmail.com \and Benjamin Fuller\\University of Connecticut\\benjamin.fuller@uconn.edu\and Alexander Russell\\University of Connecticut\\acr@uconn.edu}
\date{\today}

% Document
\begin{document}
  
  \maketitle
  % Abstract for negFE

\begin{abstract}
    We show a negative result for efficient fuzzy extractors for all distributions with fuzzy min entropy even when given a (quasi) polynomial advice string from an unbounded collaborator. 
\end{abstract}
  %!TEX root = main.tex

% Introduction for negFE
\section{Introduction}
We show a negative result for efficient fuzzy extractors for all distributions with fuzzy min entropy even when given a (quasi) polynomial advice string from an unbounded collaborator. 

We will compare to other works like \cite{fuller2016fuzzy,fuller2019continuous,fuller2020fuzzy}. 
  %!TEX root = main.tex
% Preliminary Definitions and Results for negFE

\section{Preliminaries}
In this section we will introduce existing results to help clarify our place in the literature, provide necessary existing definitions to show what we are borrowing and what we build on, and we will provide new definitions in support of our novel results.  For distributions $X, Y$, $\Delta(X, Y)$ represents the statistical distance between the two distributions.  That is, 
\[
\Delta(X, Y)\overset{def}= \frac{1}{2}\sum_{x \in X} \left| \Pr[X=x] - \Pr[Y=y]\right|.
\]

\subsection{Existing Definitions}

\begin{definition}[Entropy]
    \emph{Shannon Entropy} or simply Entropy, denoted $\ent{X}$, for some discrete random variable $X$ is a measure of how stable the outcomes of the random variable are. It is calculated as \[\ent{X} \defined \sum\limits_{i=1}^n \p{x_i}\log{\p{x_i}}\] where there are $n$ values that $X$ takes and we denote them as $x_i$. 
\end{definition}

\begin{definition}[Min Entropy]
    \emph{Min Entropy}, denoted $\minent{X}$, is a best case measure of the stability of the random variable $X$. It is calculated as \[\minent{X} \defined -\log{\max\limits_{x_i} p(x_i)}.\]  
\end{definition}

\begin{definition}[Average Conditional Min Entropy]
    \emph{Average Conditional Min Entropy}, denoted $\acminent{X}{Y}$ for two random variables $X$ and $Y$ is an average measure of the remaining entropy of the former given the outcome of the latter. It is calculated as \[ \acminent{X}{Y} \defined -\log{\Exlim{y \leftarrow Y}{\max\limits_{x} \Prob{X = x\ |\ Y = y}}}.\] 
\end{definition}

\begin{definition}[Hartley Entropy]
    \emph{Hartley Entropy} also called \emph{Hartley's Function} measures the uncertainty of a random variable in a basic way, measuring the number of outcomes the random variable has with non-zero probability. Formally, $
    \hart{X} = | \sbr{x \in X\,|\, \Prob{X = x} > 0}|.
    $
\end{definition}

\begin{definition}[Fuzzy min-entropy~\cite{fuller2020fuzzy}]

For a distribution $W$ and a distance parameter $t$, the fuzzy min-entropy of $W$, denoted $\Hfuzz(W)$ is 
\[
\Hfuzz(W) \overset{def}= -\log{ \max_{w^*} \left(\sum_{w, \dis(w, w^*)\le t} \Pr[W=w] \right)}.
\]
\end{definition}

\begin{definition}[Markov's Inequality]
    Markov's inequality is a tail bound for random variables that gives an upper bound on the probability of a random variable deviating from its mean. Let $\Pr[X>0] = 1$. Then the following inequality holds for any $\alpha > 0$: 
    \[ 
      \Prob{X \geq \alpha \cdot \Ex{X}} \leq 1/\alpha .
    \]
\end{definition}

\begin{definition}[Secure Sketch~\cite{dodis2008fuzzy}]
For metric space $(\mathcal{M}, \dis)$, a $(\mathcal{M}, \mathcal{W}, \tilde{m}, t)$-\emph{secure sketch} is a pair of algorithms $(\sketch, \rec)$ with the following properties 
\begin{enumerate} 
\itemsep0em
\item \textbf{Correctness} For all $w, w'$ such that $\dis(w, w')$, let $ss \leftarrow \sketch(w)$ then $\rec(w', ss) = w$ with probability $1$. 
\item \textbf{Security} For all distributions $W \in \mathcal{W}$ it holds that $\Hav(W | \sketch(W)) \ge \tilde{m}$.
\end{enumerate}
\end{definition}

\begin{definition}[Fuzzy Extractor~\cite{dodis2008fuzzy}]
For metric space $(\mathcal{M}, \dis)$, a $(\mathcal{M}, \mathcal{W}, \kappa, t, \epsilon)$-\emph{fuzzy extractor} is a pair of algorithms $(\gen, \rep)$ with the following properties 
\begin{enumerate} 
\itemsep0em
\item \textbf{Correctness} For all $w, w'$ such that $\dis(w, w')$, let $r, p \leftarrow \gen(w)$ then $\rep(w', p) = r$ with probability $1$. 
\item \textbf{Security} For all distributions $W \in \mathcal{W}$, let $R, P \leftarrow \gen(W)$ and $U_\kappa$ be a uniformly distributed random variable over $\zo^\kappa$ it holds that $\Delta((R, P), (U_\kappa, P))\le \epsilon.$
\end{enumerate}
\end{definition}

\subsection{Previous Results}
It has been shown that universal fuzzy extractors are impossible in the information theoretic setting. 

\subsection{Average Conditional Min-Entropy Loss}
\begin{lemma}
    \label{lem:conditionalminentloss}
    Let $\vec{X} = (X_1, X_2, \ldots, X_k)$ be independent random variables. 
    Let $Y$ be a random variable arbitrarility correlated with $\vec{X}$. 
    Then 
    \[
        \acminent{\vec{X}}{Y} \geq \sum \minent{X_i} - \hart{Y}
    \]
\end{lemma} 

\begin{proof}
    Since each $X_i$ is independent then $\minent{\vec{X}} = \sum \minent{X_i}$.
    Now, by definition,
    \begin{align*}
        \acminent{\vec{X}}{Y} &= -\log{\Exlim{y \leftarrow Y}{\max\limits_{\vec{x}} \Prob{\Vec{X} = \vec{x}\ |\ Y = y}}} \\
        &= -\log{\sum\limits_{y} \max\limits_{\vec{x}} \Prob{\vec{X} = \vec{x} \ |\ Y = y} \cdot \Prob{Y = y}}\\
        &= -\log{\sum\limits_{y} \max\limits_{\vec{x}} \Prob{\vec{X} = \vec{x} \vee Y = y}}\\
        &\geq -\log{\sum\limits_{y} \max\limits_{\vec{x}, y'} \Prob{\vec{X} = \vec{x}  ^ Y = y'}}\\
        &= -\log{2^{\hart{Y}} \cdot 2^{\minent{\vec{X},Y}}}\\
        &= \minent{\vec{X},Y} - \hart{Y}\\
        &\geq \minent{\vec{X}} - \hart{Y} \\   
        &= \sum \minent{X_i} - \hart{Y}
    \end{align*}
\end{proof}

\subsubsection{Markov Bound for Predictability}
Markov bounds are tail bounds that use Markov's Inequality to bound the probability that a random variable deviates significantly from its expected value. In Markov's inequality, we necessarily lose a multiplicative factor (here called $alpha$) in order to control the probability of the event occuring. When discussing entropy, we are dealing with a log scaled value which makes losing multiplicative factors costly. Instead, we can perform a Markov bound on the predictability scale. In this case, rather than lose a multiplicative factor in entropy, we lose a multiplicative factor in predictability which translates to a small number of bits of entropy lost for the controlled outcomes.

\todo{This is \cite[Lemma 2.2a]{dodis2008fuzzy} right? Shouldn't need to reprove.}

\begin{lemma}
    \label{lem:markovpred}
    Let $\vec{X} = (X_1, X_2, \ldots, X_k)$ be independent random variables. Let $Y$ be a random variable arbitrarility correlated with $\vec{X}$. 
    Let $\alpha > 0$, then for all but a $(1-1/\alpha)$ fraction of the $X_i$ the entropy loss is less than $\log{\alpha}/k$
\end{lemma}

\begin{proof} 

\begin{align*}
    \acminent{\vec{X}}{Y} &= \Delta\\
    -\log{\Exlim{Y}{\max\limits_{\vec{x}} \Prob{\Vec{X} = \vec{x} \,|\, Y = y}}} &= \Delta\\
    \Exlim{Y}{\max\limits_{\vec{x}} \Prob{\Vec{X} = \vec{x} \,|\, Y = y}} &= 2^{-\Delta}\\
    \Problim{Y}{\max\limits_{\vec{x}} \Prob{\Vec{X} = \vec{x} \,|\, Y = y} \geq \alpha \cdot 2^{-\Delta}} &\leq \frac{1}{\alpha} \\
    \Problim{Y}{\log{\max\limits_{\vec{x}} \Prob{\Vec{X} = \vec{x} \,|\, Y = y}} \geq \log{\alpha} -\Delta} &\leq \frac{1}{\alpha} \\
    \Problim{Y}{-\log{\max\limits_{\vec{x}} \Prob{\Vec{X} = \vec{x} \,|\, Y = y}} < \Delta -\log{\alpha}} &\leq \frac{1}{\alpha} \\
    \Problim{y\leftarrow Y}{\minent{\vec{X}\, |\, Y=y} < \Delta -\log{\alpha}} &\leq \frac{1}{\alpha}
\end{align*}
    
\end{proof}

\subsubsection{Upper bound for size of a Fuzzy Extractor}
In \cite{fuller2020fuzzy}, Fuller et al. show that the size of a fuzzy extractor can be upper bound in a general case. 
We restate their lemma here for the completeness of our main proof. 

\begin{lemma}[Lemma 5.2 in \cite{fuller2020fuzzy}]
    \label{lem:smallgeneralviable}
    Suppose $\mathcal{M}$ is $\zeroone{n}$ with the Hamming Metric, $\kappa \geq 2$, $0 \leq t \leq n/2$, and $\epsilon > 0$. 
    Suppose $(\mathsf{Gen, Rep})$ is a $(\mathcal{M,W},\kappa, t, \epsilon)$-fuzzy extractor for some distribution family $\mathcal{W}$ over $\mathcal{M}$. 
    Let $\tau = t/n$. 
    For any fixed $p$, there is a set $\mathsf{GoodKey}_p \subseteq \zeroone{\kappa}$ of size at least $2^{\kappa - 1}$ such that for every $\mathsf{key} \in \mathsf{GoodKey}_p$,
    \[
        \log{|\sbr{v \in \mathcal{M}|\prns{\mathsf{key}, p} \in \mathsf{supp}\prns{\mathsf{Gen}(v)}}|} \leq n \cdot h_2\prns{\frac{1}{2} - \tau} \leq n \cdot \prns{1 - \frac{2}{\ln{2}} \cdot \tau^{2}}, 
    \]   
    and, therefore, for any distribution $D_{\mathcal{M}}$ on $\mathcal{M}$, 
    \[
        \hart{D_{\mathcal{M}}|\mathsf{Gen}\prns{D_{\mathcal{M}}} = \prns{\mathsf{key}, p}} \leq n \cdot h_2\prns{\frac{1}{2} - \tau} \leq n \cdot \prns{1 - \frac{2}{\ln{2}} \cdot \tau^{2}}.
    \]   
\end{lemma}

\subsection{New Definitions}
Fuzzy extractors with quasipolynomial advice. 

\begin{definition}[Fuzzy Extractor with distributional advice]
Let $\mathcal{W}$ be a family of distributions indexed by $z$.  That is, one can denote each distribution in $\mathcal{W}$ as $W_Z$ which fully describes the probability mass function of $W$.  
For metric space $(\{0,1\}^n, \dis)$, a $(\{0,1\}^n, \mathcal{W}, \kappa, t, \epsilon, \ell)$-\emph{fuzzy extractor with distributional advice} is a triple of algorithms $(\gen, \rep, \aux)$ with the following properties:
\begin{enumerate} 
\itemsep0em
\item \textbf{Correctness} For all $w, w'$ such that $\dis(w, w')$, let $r, p \leftarrow \gen(w)$ then $\rep(w', p) = r$ with probability $1$. 
\item \textbf{Security} For all distributions $W_Z \in \mathcal{W}$, define $\advise \leftarrow \aux(Z)$, where $|\advise| \le \ell$, let $(R, P) \leftarrow \gen(W, \advise)$ and $U_\kappa$ be a uniformly distributed random variable over $\zo^\kappa$ it holds that $\Delta((R, P, Z), (U_\kappa, P, Z))\le \epsilon.$
\end{enumerate}
\end{definition}

\begin{definition}[Efficient Fuzzy Extractor in Known Distribution Setting]
Fix some distribution $W_Z$ Let $(\gen, \rep)$ be a $(\{0,1\}^n, \{W_Z\}, \kappa, t, \epsilon, \ell)$-fuzzy extractor.  The pair is \emph{efficient} if $\gen, \rep$ have the  additional property that they can be implemented by an algorithm that can be described in polynomial space (polynomial in the dimension $n$ of the metric space).
\end{definition}

\begin{lemma}
Let $\mathcal{W}$ be a distribution family indexed by the random family $Z$ and suppose that no $(\mathcal{M}, \mathcal{W}, \kappa, t, \epsilon, \ell)$-\emph{fuzzy extractor with distributional advice} exists for all $\ell = \poly(n)$.  Then there must be some distribution $W_Z \in \mathcal{W}$ for which no  $(\{0,1\}^n, \{W_Z\}, \kappa, t, \epsilon, \ell)$ efficient fuzzy extractor exists.
\label{lem:distributional advise suffices}
\end{lemma}
\begin{proof}[Proof of Lemma~\ref{lem:distributional advise suffices}]
We proceed by contrapositive.  Suppose that for every $W_Z\in\mathcal{W}$ there exists an efficient fuzzy extractor.  We denote these algorithms by $(\gen_Z, \rep_Z)$ respectively.  Across $Z$ let $\ell = \poly(n)$ represent the maximum space needed to implement some algorithm $\gen_Z$ or $\rep_Z$. We now describe how to build the fuzzy extractor $(\gen, \rep, \advise)$ with distributional advice.  Let $(\gen_Z, \rep_Z) \leftarrow \advise(Z)$ which has length at most $2\ell$.

Then define $\gen(x, C)$ as follows: 1) interpret $C$ as two circuits $\gen', \rep'$ and output $\gen'(x)$.  Define $\rep(x, p, C)$ interpret $C$ as two circuits $\gen', \rep'$ and output $\rep'(x', p)$.  It is clear that $(\gen, \rep, \advise)$ is a $(\mathcal{M}, \mathcal{W}, \kappa, t, \epsilon, 2\ell)$ fuzzy extractor with distributional advise.
\end{proof}


  % Sampling Section of negFE

\section{Sampling Proceedure}
The sampling proceedure that we will use and discuss in this work is two fold.
There is first, a family of distributions that we will call $\mathcal{W}$. 
There is some finite number of distributions in this family, we call $|\mathcal{W}| = \mathcal{R}$. 
We let $Z$ be an index for the distributions in the family and we denote the $Z$th distribution $W_Z$. 
The distribution $W_Z$ can then be sampled, and we denote a sample of $W_Z$ as $w_{Z} \in \zeroone{n}$. When $Z$ is clear, we will omit the subscript.

To sample a uniform point in $\mathcal{W}$, we can uniformly select $Z$ and then pick from $W_Z$ unifromly. 
This two stage process gives us more tools to reason about the ability of an efficient adversary. 

In general, we will assume that $\mathcal{W}$ is public to any adversary, algorithm, or party we may discuss. 
We will generally assume that the specific selection of $Z$ is only shared with specific parties.
We will also assume that the specific $w$ sampled from $W_Z$ is private and only known to parties who are given it explicity. 

An interesting part of this work is how we share informtaion about $W_Z$. We will allow an inefficient adversary access to the entire description of $W_Z$ and ask them to produce an advice string for an efficient adversary, that is that the advice string length is bound from above by a arbitrary polynomial function of our security parameter. 

\subsection{Distributions over $\zeroone{n}$}
\begin{enumerate}
    \item Picking $k$ points from $\zeroone{n}$ uniformly results in a distribution we will denote $U_{n,k}$. 
    Clearly, this is efficient with respect to $\max{(n,k)}$.
    The only condition here is that $k \leq 2^n$

    \item Another distribution of interest is picking $k$ points uniformly from $\zeroone{n}$ and then removing points that are within distance $t$ of one another, we denote this $U_{n,k,t}^{-}$. 
    This is also efficient, but may result in fewer than $k$ points being included in the final set. 
    The trivial bounds here also include $t \leq n$. 

    \item Another way of acheieving a simillar result is by picking points until you have $k$ (or a failure condition), that all have distance at least $t$ from one another.
    This is not guaranteed to terminate without an error condition. 
    The trivial bounds from above apply here as well. 
\end{enumerate}
 
  %!TEX root = main.tex

% Proof for NegFE

\section{Main Result}
\begin{theorem}
Fix $(\gen, \rep, \advise)$, let $n, \kappa, t, \ell $ be parameters.  Let $\mathcal{W} = \{W_Z\}$ be a family of distributions where for each $W_Z, \Hfuzz(W) \ge \alpha$.  If $(\mathcal{M,W},\kappa, t, \epsilon, \ell)$-fuzzy extractor with auxiliary input where the following conditions are satisfied:
\begin{enumerate}
\itemsep0em
\item $\kappa = \omega(\log \lambda)$ \todo{how much really}
\item $\ell = \poly(n)$ \todo{how much really}
\item Some complicated relationship with key length and distance \todo{be precise}.
\item $\alpha$ is big enough.
\end{enumerate}
Then \[
\epsilon> ....\]
\end{theorem}

\subsection{Proof Sketch}
Our proof uses the following structure
\begin{enumerate}
\item We start by using Lemma~\ref{lem:smallgeneralviable} which bounds the number of ``viable'' points for most public values $p$.  Note that this Lemma bounds the total number of points and holds even if $\gen, \rep$ have access to an arbitrary advice string.
\item We then further restrict this setting by showing that in order for an adversary to succeed on average they have to be able to align these viable points with a distribution that they have only a single sample and a polynomial length advice sting. 
\item We then argue that for large high entropy distributions this advice string can only reduce the entropy of a large fraction of viable points by a small amount. 
\item Then, we show that this small reduction of entropy for each point in the distribution means that on average the adversary cannot align the viable points with the distribution and there exists a distinguisher that can distinguish a uniform triple from a key triple. 
\end{enumerate}

\paragraph{The family $\mathcal{W}$}
We consider the following family $Z$ which chooses $k$ points with replacement from the space $\{0,1\}^n$.  That is if one samples $Z$ uniformly then one obtains the distribution $U_{n,k}$. We note there are distributions $W_Z$ in this family that have little fuzzy min-entropy.  However, we will show that this family is statistically close to a family where every distribution has fuzzy min-entropy.  For convenience we use $w_{z, 1},..., w_{z,k}$ to describe the $k$ points with nonzero probability in $W_z$ for some value $z$. Recall that points $w_{z, i}$ and $w_{z,j}$ could be equal and points are independent. 

\begin{lemma}
Let $Z$ describe a uniform sample of $U_{n,k}$.  Let $(\gen, \rep, \aux)$ be an auxiliary input fuzzy extractor.  Let $\advise \leftarrow \aux(Z)$ be of length $\ell$.  For a quad $v, p, r, \advise$ define 
\[
\mathtt{Viable}(v, p, r, \advise) = \begin{cases} 1& \Pr[\gen(v, \advise) = (r, p)]>0\\0&\text{otherwise}\end{cases}.\]
Fix some $p$ and let $\mathtt{GoodKey}$ be defined as in Lemma~\ref{lem:smallgeneralviable}. For each value $\advise$ there is some set $\mathcal{I}_{\advise}$ of size at most $\ell$, then it is true for each $i\not \in \mathcal{I}_\advise$ that
\[
\Pr\left[\mathtt{Viable}(w_i, p, \mathsf{key}, \advise) = 1\middle| \begin{aligned} z\leftarrow Z\\ \advise\leftarrow \aux(z)\\w\leftarrow W_Z\\(\mathsf{key}, p)\leftarrow \gen(w, \advise) \\\mathsf{key}\in\mathtt{GoodKey}\end{aligned} \right]< some value.
\]
\end{lemma}
\begin{proof}
Let $Z$ and $\advise$ be defined as above.
\begin{enumerate}
\item 
\end{enumerate}

\end{proof}

\subsection{Proof Setting}
Consider $\mathcal{W}$ such that each $W \in \mathcal{W}$ is a set of $2^{\phi}$ uniformly chosen independent random points in $\zeroone{n}$. 
Let $|\mathcal{W}| = r$ and let $Z \in \br{r}$ be an indexing variable for the selection of $W_Z$ from $\mathcal{W}$. 

In this proof the goal is to show that a fuzzy extractor cannot hope to hide the input point from an outside party. 
We will call the party that is building the Fuzzy Extractor the constructor and the party attempting to distiguish two settings the distinguisher. 
The Fuzzy Extractor is set up with an enrolled sample from our two stage sampling proceedure before the distiguishing game begins.
The game that the distinguisher plays is given one of two triples, real or random, do distinguish the realm to which the triple belongs. 
The real triple is the key value corresponding to the enrolled value in the fuzzy extractor, the public value produced by Gen, and the description of the distribution $W_Z$.
The random triple is the same except the key value corresponding to the enrolled value is substituted with a uniform value in the domain of the key values.

In this setting the constructor when creating the Fuzzy Extractor is aided by another party, we call this party the advisor. 
The advisor gets a full description of the distribution $W_Z$ and is allowed unbounded computation and time to produce an advice string $\mathsf{info}$. 
The advisor and constructor are unbounded in their computation and time, but the length of this string is required to be polynomial in the security parameter $\lambda$. 
The advice string is then communicated to the constructor, who also gets a sample from the distribution $w \in W_Z$. 
The constructor then is tasked with creating a Fuzzy Extractor, specifically choosing a public value $\mathsf{pub}$ which induces a partition over the space $\zeroone{n}$ and a key labeling of the partition which determines a key value for the initial sample.

\subsection{Maximum size of a Fuzzy Extractor}
From Lemma \ref{lem:smallgeneralviable}, we have the largest size of a set of viable points is $2^{\psi}$ where $\psi = n \cdot \prns{1 - \frac{2}{\ln{2}} \cdot \tau^{2}}$.

Now, it is clear from our setting that when building the Fuzzy Extractor the constructor has a single point $w$, a full description of the family of distributions $\mathcal{W}$, and the prepared advice string $\mathsf{info}$ about the specific selected distribution $W_Z$. 
We are primarily concerned with the ability of the constructor to align the points in the selected $W_Z$, with the partition induced by $\mathsf{pub}$. 
Clearly, if $\mathsf{info}$ is allowed to describe every point in $W_Z$ (or every point but $w$) then the constructor can include a point from $W_Z$ in each viable section of the partition induced by $\mathsf{pub}$. 
Fortuntely for us, the distribution has exponential entropy and the advice string has polynomial length so the advice string cannot describe the entire remainder of the distribution. 
The question remains, how much entropy remains in the distribution after seeing the advice string? 

Since each point in the distribution is independent and uniform in $\zeroone{n}$ the beginning  entropy (and min-entropy) of the distribution is $|W_Z| \cdot n$. 
Now we consider the advice string; since this string is allowed to arbitrarily depend on the distribution we can upperbound the min-entropy of the distribution conditioned on the advice string using a standard min-entropy argument found in Lemma \ref{lem:conditionalminentloss}. 
\[
    \acminent{W_Z}{\mathsf{info}} = |W_Z| \cdot 2^n - \log{|info|}
\]
  \bibliographystyle{plain}
  \bibliography{negFE}
\end{document}