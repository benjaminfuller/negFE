% Sampling Section of negFE

\section{Sampling Proceedure}
The sampling proceedure that we will use and discuss in this work is two fold.
There is first, a family of distributions that we will call $\mathcal{W}$. 
There is some finite number of distributions in this family, we call $|\mathcal{W}| = \mathcal{R}$. 
We let $Z$ be an index for the distributions in the family and we denote the $Z$th distribution $W_Z$. 
The distribution $W_Z$ can then be sampled, and we denote a sample of $W_Z$ as $w_{Z} \in \zeroone{n}$. When $Z$ is clear, we will omit the subscript.

To sample a uniform point in $\mathcal{W}$, we can uniformly select $Z$ and then pick from $W_Z$ unifromly. 
This two stage process gives us more tools to reason about the ability of an efficient adversary. 

In general, we will assume that $\mathcal{W}$ is public to any adversary, algorithm, or party we may discuss. 
We will generally assume that the specific selection of $Z$ is only shared with specific parties.
We will also assume that the specific $w$ sampled from $W_Z$ is private and only known to parties who are given it explicity. 

An interesting part of this work is how we share informtaion about $W_Z$. We will allow an inefficient adversary access to the entire description of $W_Z$ and ask them to produce an advice string for an efficient adversary, that is that the advice string length is bound from above by a arbitrary polynomial function of our security parameter. 

\subsection{Distributions over $\zeroone{n}$}
\begin{enumerate}
    \item Picking $k$ points from $\zeroone{n}$ uniformly results in a distribution we will denote $U_{n,k}$. 
    Clearly, this is efficient with respect to $\max{(n,k)}$.
    The only condition here is that $k \leq 2^n$

    \item Another distribution of interest is picking $k$ points uniformly from $\zeroone{n}$ and then removing points that are within distance $t$ of one another, we denote this $U_{n,k,t}^{-}$. 
    This is also efficient, but may result in fewer than $k$ points being included in the final set. 
    The trivial bounds here also include $t \leq n$. 

    \item Another way of acheieving a simillar result is by picking points until you have $k$ (or a failure condition), that all have distance at least $t$ from one another.
    This is not guaranteed to terminate without an error condition. 
    The trivial bounds from above apply here as well. 
\end{enumerate}
 